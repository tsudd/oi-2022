makeatletter
\def\input@path{{../../}}
\makeatother
\documentclass[../../main.tex]{subfiles}

\begin{document}

\section{Понятие о задаче целочисленного программирования}
Итак, давным давно, во время Второй Мировой Войны Европа встала
перед большой проблемой: технологический прогресс 3-ого рейха
породил таких монстров, как самолеты Люфтваффе. Перед правительством
Британии встал большой вопрос по улучшению эффективности ПВО в борьбе
с немецкими самолетами, которые доставляют большие проблемы. Еще с
1935 года Генри Тизерд работал над исследованием эффективности операций
и во время войны его наработки получили большое внимание и развитие. Тогда
был открыт исследовательский центр, где проходила работа по повышению
точности оружия при перехватке самолетов и наведении (как всегда 
война движет прогресс). В центре пытались ответить на вопрос: как из множества
решений выбрать наилучшее? Сначала решения выбирались по наитию, но затем 
появились методы по их расчету и выведению наиболее оптимальных решений (при чем
обоснованных численно). Именно так зародилось исследование операций, основной 
задачей которого является предоставление обоснованных 
данных для принятия решений.

А теперь к теории.

Задача линейного программирования является задачей условной оптимизации, где
следует найти минимум или максимум:
\begin{equation}
    \begin{cases}
        f(x_1, x_2, \dots, x_n) \to min/max \\
        f_i(x_1, x_2, \dots, x_n) \geq b_i, i = 1,2,\dots,n \\
        x_1, x_2, \dots, x_n \in \R
    \end{cases}
\end{equation}

, где $f, f_i$  --- линейные функции.

Подобным образом формулируется задача целочисленного линейного программирования
(все функции $f$ также линейны):
\begin{equation}
    \begin{cases}
        f(x_1, x_2, \dots, x_n) \to min/max \\
        f_i(x_1, x_2, \dots, x_n) \geq b_i, i = 1,2,\dots,n \\
        x_1, x_2, \dots, x_n \in \Z
    \end{cases}
\end{equation}

\end{document}